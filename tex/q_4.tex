\documentclass{amsart}
\usepackage{amsmath}
\usepackage{amssymb}
\usepackage{graphicx}
\usepackage{bm}
\usepackage[parfill]{parskip}
\usepackage{listings}
\usepackage{enumitem}
\usepackage[margin=2cm]{geometry}

\newcommand{\me}[1]{
	\title{#1}
    	\author{Abhay Shankar K: cs21btech11001}
    	\today
    	\maketitle
}
\newcommand{\bme}{
	\author{Abhay Shankar K: cs21btech11001}

	\begin{frame}
    	\titlepage 
	\end{frame}

	\begin{frame}{Outline}
    	\tableofcontents
	\end{frame}
}

\newcommand{\Me}[1]{
	\title{#1}
    \author{Abhay Shankar K}
    \maketitle
}
\newcommand{\Bme}{
	\author{Abhay Shankar K}

	\begin{frame}
    	\titlepage 
	\end{frame}

	\begin{frame}{Outline}
    	\tableofcontents
	\end{frame}
}


%LINALG
\newcommand{\cond}[2]{#1 \middle| #2}
\DeclareMathOperator*{\uni}{\exists !} % unique
\DeclareMathOperator*{\all}{\forall} % eh
\DeclareMathOperator*{\bas}{<<} % is a (ordered) basis of
\DeclareMathOperator*{\onb}{<<|} % is an orthonormal basis of
\newcommand*{\st}{\ |\ } % such that
\newcommand{\seq}[2]{\cbrak{#1_1, \ldots, #1_{#2}}} % sequence
\newcommand{\tup}[2]{\brak{#1_1, \ldots, #1_{#2}}} % tuple
\newcommand{\sqb}[2]{\sbrak{#1_1, \ldots, #1_{#2}}} % list, also matrix represented as M = [a1, a2...] with rows/columns.
\newcommand{\zseq}[2]{\cbrak{#1_0, \ldots, #1_{#2}}} % sequence
\newcommand{\ztup}[2]{\brak{#1_0, \ldots, #1_{#2}}} % tuple
\newcommand{\zsqb}[2]{\sbrak{#1_0, \ldots, #1_{#2}}} % list, also matrix represented as M = [a1, a2...] with rows/columns.
\newcommand{\pseq}[3]{\cbrak{#1_{#3}, \ldots, #1_{#2}}} % partial sequence initialy not 1
\newcommand{\iseq}[1]{\cbrak{#1_1, \ldots}} % infinite sequence
\newcommand{\izseq}[1]{\cbrak{#1_0, \ldots}} % infinite sequence
\newcommand{\rep}[3]{\ensuremath{\displaystyle \sum_{i = 1}^{#3} #2_i#1_i}} %representation as linear combination
\newcommand{\vrep}[4]{\ensuremath{\displaystyle \sum_{#4 = 1}^{#3} #2_{#4}#1_{#4}}} % variable not i
\newcommand{\prep}[4]{\ensuremath{\Sigma_{i = #4}^{#3} #2_i#1_i}} % initially not 1
\newcommand{\dr}[2]{\ensuremath{: #1 \rightarrow #2}} % domain-range
\newcommand{\con}[2]{\ensuremath{\all #1 \in #2}} %domain specs
\newcommand{\conr}[2]{\con{#1}{\sbrak{#2}}} % for natural numbers
\newcommand{\crdv}[2]{\sbrak{#1}_#2} %coordinates for vectors with a given basis.
\newcommand{\crdm}[3]{\sbrak{#1}_{#2, #3}} % for matrices
\newcommand{\nsp}[1]{\ensuremath{N\cbrak{#1}}} % nullspace
\newcommand{\nty}[1]{\ensuremath{n\cbrak{#1}}} % nullity
\newcommand{\dm}[1]{\ensuremath{dim\cbrak{#1}}} % dimension
\newcommand{\rng}[1]{\ensuremath{R\cbrak{#1}}} % range
\newcommand{\rnk}[1]{\ensuremath{r\cbrak{#1}}} % rank
\providecommand{\jor}[2]{\ensuremath{P_{#1}\brak{#2}}}
\newcommand{\spbr}[2]{\ensuremath{\brak{\cond{#1}{#2}}}}
\newcommand{\spsbr}[2]{\ensuremath{\sbrak{\cond{#1}{#2}}}}
\newcommand{\spcbr}[2]{\ensuremath{\cbrak{\cond{#1}{#2}}}}
\newcommand{\spabr}[2]{\ensuremath{\abrak{\cond{#1}{#2}}}}
\newcommand{\ip}[2]{\spbr{#1}{#2}} % inner product
\newcommand{\braket}[2]{\spabr{#1}{#2}}

%EP
\newcommand{\er}[1]{\ensuremath{\Delta#1}} % error
\newcommand{\erfrac}[2]{{\ensuremath\frac{\er{#1}}{#2}}} % eh
\newcommand{\reler}[1]{\erfrac{#1}{#1}} % relative error
\newcommand{\res}[2]{\ensuremath{\underline{#1 \mp #2}}} % result with uncertainty
\newcommand{\sci}[2]{\ensuremath{#1 \cdot 10^{#2}}} % scientific notation
\newcommand{\normpow}[3]{\ensuremath{\brak{#1^2 + #2^2}^{#3}}} % powers of standard norm
\newcommand{\e}[1]{\ensuremath{e^{#1}}} % exponent

%BEAMER
\newcommand{\bfr}[2]{\section{#1} \begin{frame}{#1} #2 \end{frame}} % beamer frames
\newcommand{\story}[3]{\bfr{}{\begin{block}{As a } #1 \end{block} \begin{block}{I want to} #2 \end{block} \begin{block}{Purpose} #3 \end{block}}} % User stories

%GVV
\newcommand{\abs}[1]{\ensuremath{\left|#1\right|}} % mod
\newcommand{\ceil}[1]{\ensuremath{\left\lceil#1\right\rceil}} % ceil
\newcommand{\floor}[1]{\ensuremath{\left\lceil#1\right\rceil}} % floor
\newcommand{\brak}[1]{\ensuremath{\left(#1\right)}} % brackets
\newcommand{\abrak}[1]{\ensuremath{\left\langle#1\right\rangle}}
\newcommand{\sbrak}[1]{\ensuremath{\left[#1\right]}}
\newcommand{\cbrak}[1]{\ensuremath{\left\{#1\right\}}}
\newcommand{\myvec}[1]{\ensuremath{\begin{pmatrix}#1\end{pmatrix}}} % matrices
\providecommand{\mn}[2]{\ensuremath{min\brak{#1, #2}}} % min
\providecommand{\mx}[2]{\ensuremath{max\brak{#1, #2}}} % max
\providecommand{\rpr}[2]{\ensuremath{P_{#1}\brak{#2}}} % random variable notation
\providecommand{\spr}[1]{\ensuremath{p\brak{#1}}} % simple notation
\providecommand{\cpr}[2]{\ensuremath{\spr{\cond{#1}{#2}}}} %conditional probability
\providecommand{\pdf}[2]{\ensuremath{p_{#2}\left(#1\right)}} % probability density notation
\providecommand{\cdf}[2]{\ensuremath{P_{#2}\left(#1\right)}} % cumulative distribution notation
\providecommand{\inv}[1]{\ensuremath{\frac{1}{#1}}} % inverse
\newcommand{\solution}{\noindent \textbf{Solution: }}
\newcommand{\question}{\noindent \textbf{Question: }}
\newcommand{\req}{\noindent \textbf{Required: }}
\providecommand{\norm}[1]{\left\lVert#1\right\rVert} % norm
\newcommand*{\permcomb}[4][0mu]{{{}^{#3}\mkern#1#2_{#4}}} % P and C
\newcommand*{\perm}[1][-3mu]{\permcomb[#1]{P}}
\newcommand*{\comb}[1][-1mu]{\permcomb[#1]{C}}
\newcommand{\ev}[1]{\mathcal{E}\sbrak{#1}}
\newcommand{\var}[1]{\textbf{Var}\sbrak{#1}}

%TOC
\newcommand{\is}[2]{\textbf{\underline{#1}}: {#2}

}
\newcommand{\si}[2]{\textit{#1}: {#2}

}
\newcommand*{\ass}{ \ensuremath{\leftarrow} }
\newcommand*{\gt}{ \ensuremath{\rightarrow} }
\newcommand{\derv}{\ensuremath{\stackrel{*}{\implies}}}
\newcommand{\sstr}[3]{\ensuremath{#1_{#2} \ldots #1_{#3}}}
\newcommand{\str}[2]{\ensuremath{#1_1 \ldots #1_{#2}}}
\newcommand*{\indist}[3]{#1 $\equiv_{#3}$ #2}
\newcommand{\pset}[1]{\ensuremath{\mathcal{P}\brak{#1}}}
\newcommand{\pdt}[3]{\ensuremath{#1, #2 \gt #3}}
\newcommand{\spc}[2]{\cbrak{#1}^{#2}}
\newcommand{\bigO}[1]{\mathcal{O}\brak{#1}}
% FOML
\newcommand{\ex}[1]{\mathbb{E}\sbrak{#1}}
\newcommand{\gauss}[1]{\mathcal{N}\brak{#1}}
\newcommand{\iter}[2]{{#1}^{\brak{#2}}}


\begin{document}
    \title{Question 4}
    \author{Abhay Shankar K: cs21btech11001 \& Kartheek Tammana: cs21btech11028}
    \maketitle

    \begin{enumerate}[label=\textbf{(\Roman*)}]
        \item 
        Knowing the maximum likelihood function, 
        \[\cpr{\mathbf{t}}{\mathbf{w}} = \prod_{n = 1}^{N} y_n^{t_n} (1 - y_n)^{t_n}\]
         we can obtain the cross-entropy error function by taking negative logarithm: 
        \begin{equation}
            E(\mathbf{w}) = - \sum_{n = 1}^{N} \brak{t_n \ln y_n + (1 - t_n) \ln (1 - y_n)} \label{eq:cross-entropy}
        \end{equation}
        
        Taking gradients of ~\ref{eq:cross-entropy} with respect to \(\mathbf{w}\), we have
        \begin{itemize}
            \item Gradient of the error: \[\nabla E(\mathbf{w}) = \sum_{n = 1}^{N} \brak{y_n - t_n} \phi_n = \mathbf{\Phi^T(y - t)}\]
            \item The Hessian: \[\nabla \nabla E(\mathbf{w}) = \sum_{n = 1}^{N} y_n (1 - y_n) \phi_n \phi_n^T = \mathbf{\Phi^T R \Phi}\] where R is the diagonal matrix given by \(R_{nn} = y_n (1 - y_n)\)
            \item Update function: \[\mathbf{w}^{(new)} = \mathbf{w}^{(old)} - \mathbf{H}^{-1} \nabla E(\mathbf{w}) = \]
            \begin{align}
                \begin{split}
                    \mathbf{w}^{(new)} &= \mathbf{w}^{(old)} - \mathbf{H}^{-1} \nabla E(\mathbf{w}) \\
                            &= \mathbf{w}^{(old)} - (\mathbf{\Phi^T R \Phi})^{-1} \mathbf{\Phi^T(y - t)} \\
                            &= (\mathbf{\Phi^T R \Phi})^{-1} \brak{\mathbf{\Phi^T R \Phi w}^{(old)} - \mathbf{\Phi^T(y - t)}} \\
                            &= (\mathbf{\Phi^T R \Phi})^{-1} \mathbf{\Phi^T R z} \label{eq:update}
                \end{split}
            \end{align}

            with \[\mathbf{z} = \mathbf{\Phi w}^{(old)} - \mathbf{R^{-1} (y - t)}\]
        \end{itemize}

        where \(\mathbf{\Phi}\) is the N × M design matrix, whose n'th row is given by \(\phi_n^T\), and \(y_n = \sigma(\mathbf{w}^T \phi_n)\).

        The algorithm for update, implemented in python:

        \begin{lstlisting}[language=Python]
import numpy as np
def update(w, Phi, t):
    y =  mp.array([sigmoid(p @ w) for p in Phi])
    R = np.diag(y * (1 - y))
    z = Phi @ w - np.linalg.inv(R) @ (y - t)
    return np.linalg.inv(Phi.T @ R @ Phi) @ Phi.T @ R @ z
        \end{lstlisting}

        \item 
        \begin{itemize}
            \item Modifying ~\ref{eq:update}, we get 
            \[(\mathbf{\Phi^T R \Phi}) \mathbf{w} = \mathbf{\Phi^T R z}\]
    
            which is the normal equation for weighted least squares. Thus, the new weight vector \(\mathbf{w}^{(new)}\) is the solution to the weighted least squares problem with \(\mathbf{z}\) and the weighing matrix \(\mathbf{R}\).
            
            \item The weighing matrix \(R\) is not constant, but depends on the parameter vector \(\mathbf{w}\).
            
            Thus, we must apply the normal equations iteratively, each time using the new weight vector \(\mathbf{w}\) to compute a revised weighing matrix \(R\). So, the algorithm is known as iterative reweighted least squares (IRLS).
        \end{itemize}
        \item 
        With the Hessian: \[\mathbf{H} = \mathbf{\Phi^T R \Phi}\]

        We know that a function is concave if it's Hessian is positive definite. Thus, it is sufficient to prove positive-definiteness of the Hessian.

        Expanding \(u^T \mathbf{H} u\), we can prove positive-definiteness \begin{align}
            \begin{split}
                u^T \mathbf{H} u &= u^T \mathbf{\Phi^T R \Phi} u \\
                &= (u^T \mathbf{\Phi^T}) \mathbf{R} (\mathbf{\Phi u}) \\
                &= \sum_{n = 1}^{N} u^T \mathbf{\phi_n}^T R_{nn} \mathbf{\phi_n u} \\
                &= \sum_{n = 1}^{N} R_{nn} \brak{\mathbf{\phi_n u}}^2 \\
                &> 0
            \end{split}
        \end{align}


    \end{enumerate}
\end{document}