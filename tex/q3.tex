\documentclass[reqno]{amsart}
\usepackage{amsmath}
\usepackage{amssymb}
\usepackage{graphicx}
\usepackage{bm}
\usepackage[parfill]{parskip}
\usepackage{listings}
\usepackage{enumitem}
\usepackage[margin=2cm]{geometry}

\newcommand{\me}[1]{
	\title{#1}
    	\author{Abhay Shankar K: cs21btech11001}
    	\today
    	\maketitle
}
\newcommand{\bme}{
	\author{Abhay Shankar K: cs21btech11001}

	\begin{frame}
    	\titlepage 
	\end{frame}

	\begin{frame}{Outline}
    	\tableofcontents
	\end{frame}
}

\newcommand{\Me}[1]{
	\title{#1}
    \author{Abhay Shankar K}
    \maketitle
}
\newcommand{\Bme}{
	\author{Abhay Shankar K}

	\begin{frame}
    	\titlepage 
	\end{frame}

	\begin{frame}{Outline}
    	\tableofcontents
	\end{frame}
}


%LINALG
\newcommand{\cond}[2]{#1 \middle| #2}
\DeclareMathOperator*{\uni}{\exists !} % unique
\DeclareMathOperator*{\all}{\forall} % eh
\DeclareMathOperator*{\bas}{<<} % is a (ordered) basis of
\DeclareMathOperator*{\onb}{<<|} % is an orthonormal basis of
\newcommand*{\st}{\ |\ } % such that
\newcommand{\seq}[2]{\cbrak{#1_1, \ldots, #1_{#2}}} % sequence
\newcommand{\tup}[2]{\brak{#1_1, \ldots, #1_{#2}}} % tuple
\newcommand{\sqb}[2]{\sbrak{#1_1, \ldots, #1_{#2}}} % list, also matrix represented as M = [a1, a2...] with rows/columns.
\newcommand{\zseq}[2]{\cbrak{#1_0, \ldots, #1_{#2}}} % sequence
\newcommand{\ztup}[2]{\brak{#1_0, \ldots, #1_{#2}}} % tuple
\newcommand{\zsqb}[2]{\sbrak{#1_0, \ldots, #1_{#2}}} % list, also matrix represented as M = [a1, a2...] with rows/columns.
\newcommand{\pseq}[3]{\cbrak{#1_{#3}, \ldots, #1_{#2}}} % partial sequence initialy not 1
\newcommand{\iseq}[1]{\cbrak{#1_1, \ldots}} % infinite sequence
\newcommand{\izseq}[1]{\cbrak{#1_0, \ldots}} % infinite sequence
\newcommand{\rep}[3]{\ensuremath{\displaystyle \sum_{i = 1}^{#3} #2_i#1_i}} %representation as linear combination
\newcommand{\vrep}[4]{\ensuremath{\displaystyle \sum_{#4 = 1}^{#3} #2_{#4}#1_{#4}}} % variable not i
\newcommand{\prep}[4]{\ensuremath{\Sigma_{i = #4}^{#3} #2_i#1_i}} % initially not 1
\newcommand{\dr}[2]{\ensuremath{: #1 \rightarrow #2}} % domain-range
\newcommand{\con}[2]{\ensuremath{\all #1 \in #2}} %domain specs
\newcommand{\conr}[2]{\con{#1}{\sbrak{#2}}} % for natural numbers
\newcommand{\crdv}[2]{\sbrak{#1}_#2} %coordinates for vectors with a given basis.
\newcommand{\crdm}[3]{\sbrak{#1}_{#2, #3}} % for matrices
\newcommand{\nsp}[1]{\ensuremath{N\cbrak{#1}}} % nullspace
\newcommand{\nty}[1]{\ensuremath{n\cbrak{#1}}} % nullity
\newcommand{\dm}[1]{\ensuremath{dim\cbrak{#1}}} % dimension
\newcommand{\rng}[1]{\ensuremath{R\cbrak{#1}}} % range
\newcommand{\rnk}[1]{\ensuremath{r\cbrak{#1}}} % rank
\providecommand{\jor}[2]{\ensuremath{P_{#1}\brak{#2}}}
\newcommand{\spbr}[2]{\ensuremath{\brak{\cond{#1}{#2}}}}
\newcommand{\spsbr}[2]{\ensuremath{\sbrak{\cond{#1}{#2}}}}
\newcommand{\spcbr}[2]{\ensuremath{\cbrak{\cond{#1}{#2}}}}
\newcommand{\spabr}[2]{\ensuremath{\abrak{\cond{#1}{#2}}}}
\newcommand{\ip}[2]{\spbr{#1}{#2}} % inner product
\newcommand{\braket}[2]{\spabr{#1}{#2}}

%EP
\newcommand{\er}[1]{\ensuremath{\Delta#1}} % error
\newcommand{\erfrac}[2]{{\ensuremath\frac{\er{#1}}{#2}}} % eh
\newcommand{\reler}[1]{\erfrac{#1}{#1}} % relative error
\newcommand{\res}[2]{\ensuremath{\underline{#1 \mp #2}}} % result with uncertainty
\newcommand{\sci}[2]{\ensuremath{#1 \cdot 10^{#2}}} % scientific notation
\newcommand{\normpow}[3]{\ensuremath{\brak{#1^2 + #2^2}^{#3}}} % powers of standard norm
\newcommand{\e}[1]{\ensuremath{e^{#1}}} % exponent

%BEAMER
\newcommand{\bfr}[2]{\section{#1} \begin{frame}{#1} #2 \end{frame}} % beamer frames
\newcommand{\story}[3]{\bfr{}{\begin{block}{As a } #1 \end{block} \begin{block}{I want to} #2 \end{block} \begin{block}{Purpose} #3 \end{block}}} % User stories

%GVV
\newcommand{\abs}[1]{\ensuremath{\left|#1\right|}} % mod
\newcommand{\ceil}[1]{\ensuremath{\left\lceil#1\right\rceil}} % ceil
\newcommand{\floor}[1]{\ensuremath{\left\lceil#1\right\rceil}} % floor
\newcommand{\brak}[1]{\ensuremath{\left(#1\right)}} % brackets
\newcommand{\abrak}[1]{\ensuremath{\left\langle#1\right\rangle}}
\newcommand{\sbrak}[1]{\ensuremath{\left[#1\right]}}
\newcommand{\cbrak}[1]{\ensuremath{\left\{#1\right\}}}
\newcommand{\myvec}[1]{\ensuremath{\begin{pmatrix}#1\end{pmatrix}}} % matrices
\providecommand{\mn}[2]{\ensuremath{min\brak{#1, #2}}} % min
\providecommand{\mx}[2]{\ensuremath{max\brak{#1, #2}}} % max
\providecommand{\rpr}[2]{\ensuremath{P_{#1}\brak{#2}}} % random variable notation
\providecommand{\spr}[1]{\ensuremath{p\brak{#1}}} % simple notation
\providecommand{\cpr}[2]{\ensuremath{\spr{\cond{#1}{#2}}}} %conditional probability
\providecommand{\pdf}[2]{\ensuremath{p_{#2}\left(#1\right)}} % probability density notation
\providecommand{\cdf}[2]{\ensuremath{P_{#2}\left(#1\right)}} % cumulative distribution notation
\providecommand{\inv}[1]{\ensuremath{\frac{1}{#1}}} % inverse
\newcommand{\solution}{\noindent \textbf{Solution: }}
\newcommand{\question}{\noindent \textbf{Question: }}
\newcommand{\req}{\noindent \textbf{Required: }}
\providecommand{\norm}[1]{\left\lVert#1\right\rVert} % norm
\newcommand*{\permcomb}[4][0mu]{{{}^{#3}\mkern#1#2_{#4}}} % P and C
\newcommand*{\perm}[1][-3mu]{\permcomb[#1]{P}}
\newcommand*{\comb}[1][-1mu]{\permcomb[#1]{C}}
\newcommand{\ev}[1]{\mathcal{E}\sbrak{#1}}
\newcommand{\var}[1]{\textbf{Var}\sbrak{#1}}

%TOC
\newcommand{\is}[2]{\textbf{\underline{#1}}: {#2}

}
\newcommand{\si}[2]{\textit{#1}: {#2}

}
\newcommand*{\ass}{ \ensuremath{\leftarrow} }
\newcommand*{\gt}{ \ensuremath{\rightarrow} }
\newcommand{\derv}{\ensuremath{\stackrel{*}{\implies}}}
\newcommand{\sstr}[3]{\ensuremath{#1_{#2} \ldots #1_{#3}}}
\newcommand{\str}[2]{\ensuremath{#1_1 \ldots #1_{#2}}}
\newcommand*{\indist}[3]{#1 $\equiv_{#3}$ #2}
\newcommand{\pset}[1]{\ensuremath{\mathcal{P}\brak{#1}}}
\newcommand{\pdt}[3]{\ensuremath{#1, #2 \gt #3}}
\newcommand{\spc}[2]{\cbrak{#1}^{#2}}
\newcommand{\bigO}[1]{\mathcal{O}\brak{#1}}
% FOML
\newcommand{\ex}[1]{\mathbb{E}\sbrak{#1}}
\newcommand{\gauss}[1]{\mathcal{N}\brak{#1}}
\newcommand{\iter}[2]{{#1}^{\brak{#2}}}


\begin{document}
    \title{Question 3}
    \author{Abhay Shankar K: cs21btech11001 \& Kartheek Tammana: cs21btech11028}
    \date{\today}
    \maketitle

    \begin{enumerate}[label=\textbf{(\Roman*)}]
        \item \question Derive the expression of likelihood and prior for a heteroscedastic setting for a single data point with input \(\mathbf{x_n}\) and output \(t_n\).
        
        \solution
        \begin{enumerate}[label=\textbf{(\alph*)}]
            \item Consider the following formula for the target variable:
            \[t = \mathbf{w^T} \boldsymbol{\phi}(\mathbf{x}) + \epsilon(\mathbf{x})\] where \(\epsilon\) is a zero mean Gaussian random variable with precision \(\beta(\mathbf{x})\) and \(\boldsymbol{\phi}\) is a deterministic function. 
            Due to heteroscedasticity, the Gaussian noise is dependent on the input \(\mathbf{x}\). 
            
            Due to the properties of Gaussian distribution, \(t\) is also normal, with its distrbution i.e. the likelihood function given by:
            \begin{equation}
                p\brak{t_n | \mathbf{x}_n, \mathbf{w}, \boldsymbol{\beta}} 
                = 
                \sqrt{\frac{\beta_n}{2\pi}} \exp \cbrak{- \frac{\beta_n}{2} \brak{t_n - \mathbf{w^T} \boldsymbol{\phi}(\mathbf{x}_n)}^2}
                \label{eq:likelihood}
            \end{equation}
    
            So, 
            \[p\brak{t_n | \mathbf{x_n}, \mathbf{w}, \boldsymbol{\beta}} = \mathcal{N} (t_n | \mathbf{w^T} \boldsymbol{\phi}(\mathbf{x}_n), \beta^{-1})\]
            \item We may assume a Gaussian prior for \(\mathbf{w}\), with arbitrary mean \(\mathbf{m_0}\) and covariance matrix \(\mathbf{S_0}\) in which case the prior is given by
        \[p(\mathbf{w}) = \mathcal{N}(\mathbf{w} | \mathbf{m_0}, \mathbf{S_0})\]

        \end{enumerate}

        \item \question Provide the expression for the objective function that you will consider for the ML and MAP estimation of the parameters considering a data set of size N.

        \solution
        \begin{enumerate}[label=\textbf{(\alph*)}]
            \item To express this more succinctly, we define the design matrix \[\boldsymbol{\Phi} = \myvec{\boldsymbol{\phi}(\mathbf{x}_1)^T \\ \vdots \\ \boldsymbol{\phi}(\mathbf{x}_n)^T} \]
            the data set \(\mathbf{X} = \seq{\mathbf{x}}{n}\)
            and the diagonal weighing matrix \(\mathbf{R}\) with \(R_{ii} = \beta_i\).
            
            Thus, the objective function for ML estimation is given by:
            \begin{align}
                \begin{split}
                    \cpr{\mathbf{t}}{\mathbf{X}, \mathbf{w}, \boldsymbol{\beta}}  
                    &= \prod_{n = 1}^{N} p\brak{t_n | \mathbf{x_n}, \mathbf{w}, \boldsymbol{\beta}} \\
                    &= \prod_{n = 1}^{N} \mathcal{N} (t_n | \mathbf{w^T} \boldsymbol{\phi}(\mathbf{x}_n), \beta_n^{-1}) \\
                    &= \exp \cbrak{- \frac{1}{2} \sum_{n = 1}^{N} \beta_n \brak{t_n - \mathbf{w^T} \boldsymbol{\phi}(\mathbf{x}_n)}^2} \\
                    &= \exp \cbrak{- \frac{1}{2} \brak{\mathbf{t} - \boldsymbol{\Phi^T w}}^T \mathbf{R} \brak{\mathbf{t} - \boldsymbol{\Phi^T w}}} \\
                    &= \mathcal{N} (\mathbf{t}|\mathbf{\Phi w, R^{-1}}) \label{eq:ml} 
                \end{split}
                % \equiv \ln p &= \text{constant} + \frac{1}{2} \sum_{n = 1}^{N} \beta_n \brak{t_n - y(\mathbf{x_n, w})}^2 
            \end{align}
            \item 
                Given
                    \[p(\mathbf{w}) = \mathcal{N}(w|\mathbf{m_0, S_0})\]
                and
                    \[p(\mathbf{t|w}) = \mathcal{N}(\mathbf{\Phi w, R^{-1}})\]

                Let \(\mathbf{z} = \myvec{\mathbf{w} \\ \mathbf{t}}\) as follows:
                \begin{align}
                    \begin{split}
                        \ln p(\mathbf{z}) &= \ln p(\mathbf{w}) + \ln p(\mathbf{t|w}) \\
                            &= -\inv{2} (\mathbf{w - m_0})^T \mathbf{S_0}^{-1} (\mathbf{w - m}_0) \\
                            &\ - \inv{2} (\mathbf{t - \Phi w})^T \mathbf{R} (\mathbf{t - \Phi w}) + \text{constant} \label{eq:map1}
                    \end{split}
                \end{align}

                Furthermore, due to linearity of expectation, we have 
                \[\ev{\mathbf{z}} = \myvec{\ev{\mathbf{w}} \\ \ev{\mathbf{t}}}\]
                from which
                \[\cov{\mathbf{z}} = \myvec{
                    \var{\mathbf{w}} & \cov{\mathbf{w}, \mathbf{t}} \\
                    \cov{\mathbf{t}, \mathbf{w}} & \var{\mathbf{t}}
                }\]

                From ~\eqref{eq:map1}, it is clear that \(p(\mathbf{z})\) is a Gaussian distribution. Now we complete the square.
                
                To find the covariance of \(\mathbf{w|t}\), we consider the single term of second order in \(\mathbf{w}\) from ~\eqref{eq:map1}: 

                \[\inv{2} \mathbf{w^T \Sigma}^{-1}\mathbf{w} = \inv{2} \mathbf{w^T}(\mathbf{S_0}^{-1} + \mathbf{\Phi R \Phi^T})\mathbf{w} \]

                We treat \(\mathbf{t}\) as a constant.

                Thus, the covariance is given by \[\mathbf{\Sigma} = (\mathbf{S_0}^{-1} + \mathbf{\Phi^T R \Phi})^{-1}\]

                Similarly, we may obtain \(\boldsymbol{\mu}\) using the terms of ~\eqref{eq:map1} of first order in \(\mathbf{w}\). We have

                \[\mathbf{w^T \Sigma}^{-1} \boldsymbol{\mu} = \mathbf{w^T} \mathbf{S_0}^{-1} \mathbf{m_0} + \mathbf{w^T \Phi^T R t}\]
                    which yields
                \[\boldsymbol{\mu} = \boldsymbol{\Sigma} (\mathbf{S_0}^{-1} \mathbf{m_0} + \mathbf{\Phi^T R t})\]

                Thus, the MAP objective function is \[p(\mathbf{w|t}) = \mathcal{N} (\mathbf{w|\boldsymbol{\mu}, \Sigma})\]
                
        \end{enumerate}

        

        \item \question Show \[E_\mathcal{D} = \sum_{n = 1}^{N} r_n \cbrak{t_n - \mathbf{w^T \phi(x_n)}}^2 \] and find \(\mathbf{w}\) that minimizes \(E_\mathcal{D}\).
        \solution
        \begin{enumerate}[label=\textbf{(\alph*)}]
            \item Taking logarithm of ~\eqref{eq:likelihood}
            \[-\ln p(\mathcal{D} | \mathbf{w}) = \frac{N}{2} \ln \beta_n - \frac{N}{2} \ln (2\pi) + E_\mathcal{D}\]
            with
            \begin{equation}
                E_\mathcal{D} = \inv{2} \sum_{n = 1}^{N} \beta_n \cbrak{t_n - \mathbf{w^T \phi(x_n)}}^2 \label{eq:ED}
            \end{equation}
    
            Setting \(r_n = \frac{\beta_n}{2}\), we obtain the desired equation.

            \item Now, we may obtain the \(\mathbf{w}\) that minimizes \(E_\mathcal{D}\) by differentiating ~\eqref{eq:ED} and setting the derivative to zero like so:
        \[\sum_{n = 1}^{N} \beta_n \cbrak{t_n - \mathbf{w^T \phi(x_n)}} \mathbf{\phi\brak{x_n}^T} = 0\]
        Whence 
        \[ \sum_{n = 1}^{N} t_n \beta_n \mathbf{\phi\brak{x_n}}^T = \mathbf{w^T}\sum_{n = 1}^{N} \beta_n \mathbf{\phi\brak{x_n}\phi\brak{x_n}^T}\]

        Using the matrix \(\mathbf{R}\) defined as before, taking the transpose of both sides yields
        \begin{align}
            \begin{split}
                \mathbf{\Phi^T R t} &= \mathbf{(\Phi^T R \Phi) w} \\
                \implies \mathbf{w_{ML}} &= (\mathbf{\Phi^T R \Phi})^{-1} \mathbf{\Phi^T R t}
            \end{split}
        \end{align}
        \end{enumerate}
        
    \end{enumerate}


\end{document}